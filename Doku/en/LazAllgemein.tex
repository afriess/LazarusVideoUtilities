\subsection{Einleitung}
\gls{DirectShow} (ehemals \"Direct Media\")\footnote{Quelle: Wikipedia\cite{400}}
DirectShow oder ehemals ActiveMovie bzw. DirectX Media dient der Verarbeitung von Video- und Audio-Dateien, womit sich verschiedenste Arten von Video-Dateien (AVI, MPEG) und Ton-Dateien (z. B. MP3) wiedergeben oder erstellen lassen. Es wird auch Streaming unterst�tzt und ist durch DirectShow-Filter beliebig erweiterbar.
DirectShow wurde inzwischen aus dem DirectX SDK entfernt und ist in das Windows Plattform-SDK aufgenommen worden. Somit geh�rt DirectShow streng genommen nicht mehr zu DirectX, sondern ist jetzt ein Bestandteil der Windows-Plattform.

Lazarus und Freepascal unterst�tzen eine vielzahl von Umgebungen, DirectShow ist eine FrameWork nur f�r Windows 32 und 64. Somit k�nnen die folgenden Informationen nur f�r Windowsversionen gelten auf denen zumindest DirectX9 oder neuer installiert ist. 

Die meisten modernen Ger�te haben bereits eine Kamera eingebaut. Diese ist intern meistens als USB-Device angeschlossen. Wie sie genau angeschlossen ist, ist f�r den Programmiere nicht relevant, solange die Kamera vom System erkannt wird und die entsprechenden Treiber von Windows geladen werden. Entweder sind die Treiber bereits generisch in Windows enthalten oder werden vom Hersteller zur Verf�gung gestellt.   

In den Beispielen wird es spezielle um Informationen gehen die sich in erster Linie mit Kameras und den verbunden Einstellungen besch�ftigt.

\subsection{Abfragen der Kategorien}
Die verschiedenen Filter sind in Kategorien eingeteilt, diese Ergeben sich aus dem Zweck der Filter und sind vom System her vorgegeben. Kategorien werden am einfachsten �ber \textbf{TSysDevEnum} aus dem Paket \textbf{DXSUtils} abgefragt. Wir lassen uns hier den lesbaren Namen \gls{FriendlyName} statt der \gls{GUID} anzeigen. Sp�ter k�nnen wir die weiteren Informationen die wir �ber \textbf{TSysDevEnum} abfragen, durch die Angabe einer \gls{GUID} auf die ben�tigten einschr�nken. 

\begin{verbatim}
unit BasicEnumCatMain;

{$mode objfpc}{$H+}

interface

uses
  Classes, SysUtils, FileUtil, Forms, Controls, Graphics, Dialogs, StdCtrls;

type

 {TBasicEnumCatForm }

TBasicEnumCatForm = class(TForm)
  BuEnum: TButton;
  LbCategories: TListBox;
  procedure BuEnumClick(Sender: TObject);
end;

var
  BasicEnumCatForm: TBasicEnumCatForm;

implementation

uses
  DXSUtil, DirectShow9;

{$R *.lfm}

  {TBasicEnumCatForm }

procedure TBasicEnumCatForm.BuEnumClick(Sender: TObject);
var
  i: integer;
  SysDev: TSysDevEnum;
begin
  SysDev:= TSysDevEnum.Create;
  try
    if SysDev.CountCategories > 0 then
      for i := 0 to SysDev.CountCategories - 1 do begin
        LbCategories.Items.Add(SysDev.Categories[i].FriendlyName);
      end;
  finally
    FreeAndNil(SysDev);
  end;
end;
end.
\end{verbatim}
Das Ergebinis siehtfolgendermassen aus:

